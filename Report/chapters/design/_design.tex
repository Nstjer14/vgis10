\graphicspath{{figures/design/}}
\chapter{Design}\label{ch:design}
Based on the acquired knowledge from \autoref{ch:related}, \autoref{ch:analysis}, and \autoref{ch:data_anal} a design of an occlusion detection solution is made. This is implemented using Python3 with the Keras API working with Tensorflow.

This chapter gives an in-depth description of how the implementations are made and the theory behind them.
\section{Faster R-CNN}
To be able to detect objects in a frame using a \gls{cnn} the Faster R-\gls{cnn} proposed by \cite{frcnn} is used. The chosen solution model is the third iteration of the Region based Convolutional Neural Network by \cite{rcnn}. As the title proposes, the network is a region proposal based network, used for object detection in an image.

\subsection{Object Detection}
Object detection opposed to image classification aims to locate defined objects or instances in images, and often in images containing several objects to locate at the same time, whereas an image classification solution aims to classify one object in an image and recognise the the single object without any localisation.

\subsection{Region-based Convolutional Neural Network (R-CNN)} 
The model for the R-\gls{cnn} begins with a region search and then do the classification. This model uses \textit{Selective Search} which initialises regions in the image and then merge these with a hierarchical grouping. The final grouping is then a box containing the entire image. The regions are grouped in relation to colour space and and similarity.

\section{Region Proposal}

\section{Intersection over Union}
