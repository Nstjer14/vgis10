\chapter{Conclusion}\label{ch:conclusion}\glsresetall
Throughout the project, the aim has been to find a solution to the problem statement:

\textit{How can a zebrafish occlusion detection solution be implemented in order to potentially optimise zebrafish tracking?}\\

In order to be able to detect zebrafish occlusions two different solutions are made. To be able to cater to different occlusion handling approaches a simple image classification occlusion detection and a more complex multi-class object occlusion detection were developed. The simple image classification would require a user to solve occlusions in a tracking system. While the more complex object detection solution system will enable application of fit solutions based on the category automatically and thereby require no user interaction during tracking.\\

The image classification achieves a validation accuracy of $93\%$ when trained for 15 epochs, but shows a lot of instability which can indicate a coincidence. When training for longer, no convergence is achieved which may indicate the need of more data or tweaking of the hyper parameters.\\

The object detection achieves a \gls{map} of $66.8\%$, and is able to locate both occlusions and single zebrafish as \textit{no occlusion}. It is not able to detect all types of occlusions with some classes having a rather low average precision in testing indicating the need for more data to increase performance.\\

The report proposes two different solutions on how a zebrafish occlusion detection solution can be made, and presents a novel categorisation of multiple zebrafish occlusions. The multi-class object detection also serves as a proof of concept for the implementation of this solution in a complete tracking system.