\chapter{Introduction}\label{ch:intro}
The Zebra fish (Danio Rerio) is being used for scientific research to study human diseases and behavioural patterns in multiple different scenarios. The use of the Zebra fish is due to a lot of physiological similarities to humans together with a 70\% human disease genes similarity, which means results from experiments on the fish can be transferable to humans \citep{RahmanKhan2018, Delcourt2018}.

Behavioural analysis of the fish are mainly done using a video tracking system to record multiple different analytics of each individual in an arena while being the least invasive as possible \citep{Levin2009, Delcourt2018}. \todo{add more here}

When tracking, it is crucial to be able to keep track of each fish in the aquarium individually. Issues with tracking can occur if a fish disappears in the camera view, that can lead to loss in data which is undesirable. In a fish tank with a lot of water, the fish are able to swim above one another, which can lead to one fish covering the other in the picture. These coverings called occlusions is a general issue when tracking fish in an aquarium in which the fish are allowed to swim freely laterally.

The handling of occlusions is done in multiple ways and rely on several different methods of solving the issue. Where some solutions rely on tracking the centre of mass others focus on tracking the head of the fish or fingerprinting each individual in the aquarium, but they all rely on a high contrast between the the fish and the background \citep{Delcourt2018}. The solutions often perform well when no or few occlusions occur, but lack performance in maintaining individual IDs of the fish the more occlusions there are \citep{Delcourt2018}.

In an attempt to help solve this issue an analysis of different types of occlusions is carried out. This is done to investigate if different types of occlusions require different solutions in handling them. \cite{Romero-Ferrero2019} have trained a neural network to be able to detect the occlusions occurring in an image, but have not split the occlusions into different types. \todo{Lav en afrunding af afsnittet som skal lede videre til related work.}