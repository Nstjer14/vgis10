\chapter{Introduction}\label{ch:intro}
In biology and medicine, tests and experiments are often performed on animals before any application on humans are done, both due to ethics, feasibility, and possible risks. This type of animals are known as model organisms, as they often have some biological resemblance to humans. An example of a model organism are mice, which are often used to study medical treatments for humans \citep{Perlman2016, RahmanKhan2018}.

However, due to their size and applicability the zebrafish (Danio Rerio) are also used as a model organism. As the embryo of the fish is clear, early development research is easy to observe. Furthermore the fish and humans share a lot of pathways which control development of central nervous systems and about $70\%$ of genes associated with diseases \citep{RahmanKhan2018}.\\

As the fish are highly social animals, they are also used to study social phenotypes and behavioural patterns when affected by drugs or pheromones. By studying the shoaling behaviours under influence of different substances, the drug’s effect on social behaviour is investigated.

To be able to recognise differences in behaviour, each fish must be tracked individually over time. The tracking is often done using tracking systems, as manually annotating the data often proves infeasible. According to \cite[Green2012] the use of a tracking system performs as effectively as manual annotations, but in a faster manner. They do state that the use of a tracking system is not without some complications due to occurring occlusions of an individual. These occlusions often occur due to their social behaviour, which can lead to a loss of data in the trajectories of each fish, breaking them into tracklets, which then needs to be connected again.

Occlusions can also lead to the loss of a unique ID belonging to a fish, which will then need re-identification either automatically or manually. If reassignment of the ID is not done, the individual will get a new ID \citep{Feijo2018}.\\

According to \cite{Qian2017} occlusions occur both while recording an aquarium from above and from the side, and states that more occlusions occur when recording the side of the aquarium due the shape of the fish, often being higher in a side view than wide in the top view.

An occlusion can cause issues for one type of tracking system while others are able to maintain a trajectory during some occlusions, this is dependent on the the type of detection of the individual fish. When detecting and tracking the head of the fish, some occlusions are solved when the head of both fish are visible, whereas other solutions employing a centre of mass detection may only detect one fish instead of two.

Solutions to missing data, due to occlusions, are often relinking tracklets to create complete trajectories. By computing a state vector for each individual and using a Kalman filter, predictions of the fish’s position are made. This is used to determine which fish belongs to the detections made after an occlusion has occurred \citep{Feijo2018, Qian2014}. 

Instead of handling occlusions in a tracking step of a system it may be feasible to do it while detecting the fish in each frame. By doing this, more actions can be performed towards solving the occlusion instead of patching in missing data. Both \cite{Romero-Ferrero2019} and \cite{Dolado2015} propose solutions which detect occlusions before tracking in an effort to solve them using computer vision.\\

REWRITE THIS
But does different kinds of occlusions require different kinds of solving? Is some occlusions easier to handle than others?
REWRITE THIS


%The Zebrafish (Danio Rerio) is being used for scientific research to study human diseases and behavioural patterns in multiple different scenarios. The use of the Zebrafish is due to a lot of physiological similarities to humans together with a 70\% human disease genes similarity, which means results from experiments on the fish can be transferable to humans \citep{RahmanKhan2018, Delcourt2018}.
%
%Behavioural analysis of the fish are mainly done using a video tracking system to record multiple different analytics of each individual in an arena while being the least invasive as possible \citep{Levin2009, Delcourt2018}. \todo{add more here}
%
%When tracking, it is crucial to be able to keep track of each fish in the aquarium individually. Issues with tracking can occur if a fish disappears in the camera view, that can lead to loss in data which is undesirable. In a fish tank with a lot of water, the fish are able to swim above one another, which can lead to one fish covering the other in the picture. These coverings called occlusions is a general issue when tracking fish in an aquarium in which the fish are allowed to swim freely laterally.
%
%The handling of occlusions is done in multiple ways and rely on several different methods of solving the issue. Where some solutions rely on tracking the centre of mass others focus on tracking the head of the fish or fingerprinting each individual in the aquarium, but they all rely on a high contrast between the the fish and the background \citep{Delcourt2018}. The solutions often perform well when no or few occlusions occur, but lack performance in maintaining individual IDs of the fish the more occlusions there are \citep{Delcourt2018}.
%
%In an attempt to help solve this issue an analysis of different types of occlusions is carried out. This is done to investigate if different types of occlusions require different solutions in handling them. \cite{Romero-Ferrero2019} have trained a neural network to be able to detect the occlusions occurring in an image, but have not split the occlusions into different types. \todo{Lav en afrunding af afsnittet som skal lede videre til related work.}