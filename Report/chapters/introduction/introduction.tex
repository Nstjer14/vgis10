\chapter{Introduction}\label{ch:intro}
Loligo Systems is a research company who specialises in developing products for aquatic biology, animal physiology, and behavioural research and teaching. The behavioural analysis consists of several different products, depending on the application area. A video-based tracking called \textit{LoliTrack} is introduced to enable a user to analyse the behaviour of multiple individual animals such as zebra fish, rodents, or insect swarms. Some of the parameters collected with the tracker are: activity, velocity, direction of movement, orientation, and more.

This type of behavioural analysis is often used to compare a clean and healthy environment with a contaminated one, to analyse the effects of the exposure an individual or a group may suffer from. For this kind of experiment the zebra fish is often used opposed to mice. Even though mice a more similar to humans, when testing for an experiment's effect on the human body, zebra fish can prove easier to both maintain and cheaper to host. The zebra fish resembles the human genes at close to $ 70\% $ and the internal body structure has a lot of similarities as well.

\textit{LoliTrack} only works with a single camera, which means the patterns are limited to two dimensions on a single plane. This can introduce both occlusions and somewhat faulty data, if the fish are swimming in an aquarium where they are able to swim freely in all three dimensions.

Introducing a third dimension to the data can increase precision of trajectories and may be able to prevent occlusions by fish swimming above one another.
\todo{Needs more introduction of the tracking in general - it just pops up here...}